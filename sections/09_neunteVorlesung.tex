\section{Vorlesung vom 16.04.2019}
In der heutigen Vorlesung war ich nicht anwesend. Trotzdem werde ich kurz ein paar, für mich relevanten, Themen im Lerntagebuch reflektieren.\\

Das Hauptthema war \textit{Der verantwortungsvolle Umgang mit technologischen Innovationen und Entwicklungen}. Dabei stellen sich schnell einige ethische Fragen bezüglich des enormen technologischen Fortschritts.\\

\begin{itemize}
	\item [ ] Wie sollen wir mit neuen technologischen Entwicklungen
umgehen?\\
	\item [ ] Welche Folgen haben sie – für uns als Einzelne und als
Gesellschaft?\\
	\item [ ] Wie können wir z.B. die Digitalisierung der Arbeitswelt
verantwortungsvoll gestalten?\\
\end{itemize}

Diese Fragen können nicht einfach so beantwortet werden, weil sie abhängig sind von der Ethik und Moral der Individuen. Zudem stellt sich auch das Problem, dass wenn die Digitalisierung in allen Berufsdomänen der Arbeitswelt weiter voranschreitet, steigt die Arbeitslosenzahl ins unermessliche. Dies würde bedeuten, dass das Volk auch kein Geld mehr verdient. Was würde dann geschehen?\\

Die Ethik\footnote{Orientierungswissenschaft} beschäftigt sich grundsätzlich mit der Fragen nach dem guten und gerechten Leben. Die Moral mit der Summe der Traditionen und Werten die in einer Gesellschaft gelebt und hochgehalten werden. Diese Grundsätze können religiös fundiert sein, müssen es aber nicht.\\

Leider muss ich bei gewissen Aussagen auf den Folien intervenieren. Dass...
\begin{itemize}
	\item[ ] Maschinen die Frage über Leben und Tod \textbf{nur} so behandeln können, wie die Frage ob es warm oder kalt im Raum ist,\\
	\item[ ] künstliche Intelligenz sich \textbf{nicht} fragt, ob ein Krieg gerechtfertigt ist oder ob es in einer konkreten Situation Alternativen zu einer tödlichen Aktion gibt,\\
	\item[ ] Maschinen \textbf{kein} Gewissen haben, \textbf{kein} Mitleid empfinden, töten, \textbf{ohne} den
Wert des Lebens zu anerkennen und für ihre Handlungen keine Verantwortung übernehmen,\\
\end{itemize}
sind gänzlich kontroverse Meinungen zum technologische Fortschritt und bilden eine Analogie zu der Filmreihe \textit{Terminator}\footnote{ich möchte hier lediglich eine neutrale Meinung dazu äussern}. Maschinen sind rein deterministisch funktionierende Systeme. Somit kann in der Entwicklung dieser auf genau jene Dinge reagiert werden, damit auch sie allenfalls nach einer bestimmten Moral funktionieren. Zudem stelle ich mir die Frage, wenn wir Menschen tatsächlich eine echte, \textbf{selbst denkende} und lernfähige \textit{Intelligenz} \glqq erschaffen\grqq\;(nicht eine dieser heutigen pseudo KI's in Handys und anderen Geräten), würde diese dann nicht auch beginnen, sich dieselben essenziellen Fragen über deren Existenz und Sinn des Daseins stellen? Denn wäre dies der Fall, was meiner Meinung nach möglich sein könnte, dann wären diese Maschinen sehr wohl in der Lage, anders auf die oben zitierten Aussagen zu reagieren. Wenn nicht sogar in menschlicher Art und Weise.\\