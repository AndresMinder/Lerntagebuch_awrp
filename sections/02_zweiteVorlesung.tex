\section{Vorlesung vom 26.02.2019}
Das Hauptthema dieser Vorlesung galt dem Judentum. Es wurde erläutert, dass es unterschiedliche Auffassungen, rsp. Interpretationen der Religion gibt. Um diese auch etwas besser verstehen zu können, soll jeder selbst etwas aus seinem Kreis herausstehen und auch sich selbst in kritischer Perspektive reflektieren und offen für Anderes/Neues sein. Zusätzlich wurde anhand von Mayim Bialik ein Beispiel gezeigt, wie die Religion und die Wissenschaft im Weltbild eines Menschen koexistieren können. In einigen Punkten stimmen ihren Vorstellungen und meine überein. Zum Beispiel, dass der Mensch mehr als nur Materie ist, bezweifle ich auch, sondern ich assoziiere jedem individuellem Lebewesen auch einen spirituellen Teil. Zudem glaube ich, dass alles auf einer bestimmten Weise spirituell verbunden ist. Denn nur durch seine Transzendenz, muss dies nicht heißen, dass es nicht da ist. Sie hat allerdings klar eine monotheistische Vorstellung, wobei ihr Gott metaphorisch auch wirklich existiert. Meine Ansichten sind da eher agnostizistischer Natur.\\

Im heutigen Judentum gibt es drei Gruppierungen. Das \textit{Reformjudentum}, bei dem aufklärerisches und liberales Gedankengut im Bestreben, mit dem modernen Leben nicht in einen Konflikt zu geraten. Das \textit{orthodoxe Judentum}, wo in Abgrenzung zur Aufklärung und Aufrechterhaltung der Tora als göttlich inspirierte Schrift strikt festzuhalten ist. Das \textit{konservative Judentum}, welches eine Mittelstellung ist, bei der die Gebote im Auge behalten, aber doch auch Anpassungen ans moderne Leben unternommen werden.\\

Des Weiteren wurden einige Begriffe kurz erläutert:\\


\subsection*{Monotheismus}
	Der Monotheismus ist das Charakteristikum der drei abrahamitischen, oder eben 			monotheistischen Religionen von Judentum, Christentum und Islam. Der Welt 				liegt dabei ein einziges, einheitsstiftendes Prinzip zugrunde, bei 						dem es um ein personales \textit{Du} mit einem ethischen und rituellem 					Anspruch geht. Zusätzlich gilt \textit{Gott} als ein transzendentes 					anthropomorphes Wesen. \\
\subsection*{Schriften und Feste}
	\begin{itemize}
		\item Tanach (TNK):\\
			Dies gilt als die schriftliche Überlieferung.\\
			\begin{itemize}
				\item Tora (Weisung) = 5 bücher Mose: Schöpfung - Sinflut - 							Abraham, Isaak, Jakob (Israel) - Auszug aus Ägypten - Offenbarung 						am Sinai\\
				\item Nevi'im (Propheten)=Prophetenbücher wie Buch Jesaja\\
				\item Ketubim (Schriften)\\
			\end{itemize}
			\item Talmud (Babylonischer und Jerusalemer):\\
				Der Talmud gilt als die verschriftlichte mündliche Überlieferung, in 					der die Debatten der Auslegungsmöglichkeiten der Tora zusammengefasst 					sind.\\
	\end{itemize}
\subsection*{Tora (ein erster Teil der heiligen Schrift)}
Die Schriftrolle der Tora gilt als das heiligste Objekt im Judentum und wird nur von Hand abgeschrieben. Im Kern dieser Schriftrolle stehen die zehn Gebote, wie jene die auch in der Bibel der Christen stehen (einen Verweis auf den Dekalog). Zudem erwähnenswert ist, dass es im Judentum nicht so sehr um rechten Glauben, sondern um rechte Taten, welche durch Gebote und Verbote strukturiert sind geht.\\