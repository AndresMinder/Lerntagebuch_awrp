\section{Vorlesung vom 26.03.2019}
Bei der heutigen Vorlesung hatten Christoph Urich, Marius Hasler und ich unsere Präsentation. Mein Teil bezog sich hauptsächlich auf ein Textstück von dem berühmten Philosophen Immanuel Kant, welcher sich mit der \textit{aufgeklärten Welt} auseinandersetzte. Allerdings möchte ich mehr auf den im Unterricht behandelten Stoff und nicht auf die Präsentation eingehen.\\

Das behandelte Thema passte zu unserer Präsentation und befasste sich mit:\\
\begin{itemize}
	\centering
	\item[ ] \glqq \textit{Was ist Aufklärung?}\textit{ Was ist Vernunft?}\grqq\\
\end{itemize}

Für die \textit{Vernunft} existiert das klassische Model. \glqq Der Mensch besitzt Vernunft\grqq. Hier wird die Vernunft als wesentlicher Bestandteil der menschlichen Seele gesehen. Aus \textit{vernünftig sein} wird geschlossen, dass die Dinge beweisbar wahr sein müssen. Es wird ein wissenschaftlich geführter Beweis benötigt, um etwas als Wahr zu bekennen. Hier ist noch wichtig zu erwähnen, dass bei der wissenschaftlicher Beweisführung die Reproduzierbarkeit eine zentrale Rolle spielt. Wo für mich auch ein kleines Problem/Widerspruch entsteht, denn einige Dinge sind nicht immer reproduzierbar. \\

Das zeitgenössische Model widerspiegelt die Vernunft so, dass im Menschen selbst die Vernunft angelegt ist. Damit ist gemeint, dass der Mensch lediglich die Möglichkeit zu vernünftigen Denken und Handeln hat. Was bedeutet, dass der Mensch zwangsläufig nicht immer vernünftig handelt. Allerdings gibt es auch hier einige Widersprüche. Nämlich teilen nicht alle Völker, Länder und sonstige Gruppierungen dieselbe Ethik/Moral. Was uns dann zum Problem führt, das \textit{vernünftige Handeln} für alle gleich zu definieren. Offensichtlich ist dies nicht möglich, denn nicht einmal in der Nahrungsaufnahme kann dies einheitlich betrachtet werden. Zum Beispiel sind für Hindus die Kühe heilig und es wäre äusserst unvernünftig für sie, eine Kuh zu verletzen, geschweige denn, sie zu essen. In den westlichen Ländern wie hier in der Schweiz ist dies allerdings kein Problem. Dies führt mich selbst zum Schluss, dass nur der Begriff \textit{Vernunft} definiert werden kann, nicht aber ob ein Mensch vernünftig handelt oder nicht. Dazu müsste die Moral mit der dieser Mensch aufwuchs und auch die Erfahrungswerte desjenigen berücksichtigt werden, um zu urteilen.\\

Im weiterführenden Unterricht wurde eine \textit{Revision} des klassischen Vernunftverständnisses erläutert. Dafür möchte ich gerne die Aussage \glqq \textit{Es ist unmöglich, durch Beweise zu absolut sicherem Wissen zu gelangen}\grqq\;aufgreifen. In der wissenschaftlichen Beweisführung wird immer mit Modellen gearbeitet. In diesen Modellen werden nur die wichtigsten Grössen berücksichtigt, wie auch deren Bedingungen. Es kann also nicht möglich sein zu absolut sicherem Wissen zu gelangen, wenn Modelle zum vereinfachten Verständnis der Dinge mit bestimmten Bedingungen benutzt werden\footnote{zu erwähnen ist, dass hier das Schlüsselwort \textbf{absolut} ist}. Dazu müsste alles zu allen Bedingungen berücksichtigt werden.\\

Für die Vernunft wurde ein Fazit festgehalten. \\
\begin{itemize}
	\item Es gibt keine zweifelsfreie absolute Wahrheit\\
	\item Vernünftiges Denken $\rightarrow$ Annäherung an die Wirklichkeit\\
	\item Überzeugungen können wahr sein, aber es ist genauso gut möglich, dass wir uns irren\\	
\end{itemize}
Mit diesem Fazit stimme ich überein. Die Gründe, weshalb ich dies auch so sehe, habe ich oben in den Absätzen bereits vermerkt.\\