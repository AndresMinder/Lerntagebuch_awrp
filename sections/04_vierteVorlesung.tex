\section{Vorlesung vom 12.03.2019}
Diese Vorlesung wurde mit einer einfachen, aber doch recht komplexen Frage eingeläutet:\\

\begin{center}
	\textbf{\large{\glqq Was ist Religion?\grqq}}
\end{center}

In der heutigen Zeit ist Religion ein recht kontroverses Thema. Wie offen darf man seine Religion in der Öffentlichkeit zeigen und was ist mit dem Begriff \textit{Religion} eigentlich gemeint? Mit der Definition der Religion gibt es da Probleme einen einheitlichen Begriff zu definieren. Der Ursprung des Begriffs könnte bei \textit{relegere} liegen, was sich etwas immer wieder zuwenden bedeutet. Übertragen auf die Religion ist damit ein genaues Einhalten von religiösen Handlungen wie Gebete oder Rituale gemeint. Andernfalls könnte der Begriff auch von \textit{religare} abstammen, was soviel wie verbinden oder anbinden umschreibt. Hier wird das Verbunden-Sein des Menschen mit einer transzendenten Wirklichkeit assoziiert. \\

In einer Gruppenarbeit mussten wir uns einer Definition von verschiedenen bekannten Persönlichkeiten zuordnen\footnote{es waren sechs verschiedene Definitionen}. Ich entschied mich dabei für die folgende Definition:\\

\begin{center}
	\textit{\glqq Religion ist der Versuch, nichts in der Welt als fremd, 						menschenfeindlich, schicksalhaft, sinnlos anzunehmen, sondern alles, was 					begegnet, zu verwandeln, einzubeziehen in die eigene menschliche Welt.\grqq}
\end{center}
\begin{flushright}
	DOROTHEE SÖLLE\\
\end{flushright}

Dies ist eine recht weltoffene Definition, welche die Religion selbst nur als Versuch für die Einbindung des Umfelds in die eigene Welt des Individuums beschreibt. Die Religion soll dem Leben und dem was darin passiert eine tiefere Bedeutung, einen Sinn zuteilen. Es ist aber auch zu erkennen, dass die Verfasserin des Zitats auch etwas auf die Nächstenliebe baute. Dies zeigt sich in der Aussage \textit{\glqq [...] nichts in der Welt als fremd, menschenfeindlich, [...] anzunehmen, [...]\grqq}. Daraus lässt sich schließen, dass Dorothee Sölle vermutlich eine Christin war. Nach kurzer Recherche ergab sich auch, dass sie evangelische Theologin war. Daraus bestätigt sich ihre christliche Weltansicht.\\

Als kleines Fazit lässt sich somit sagen, dass es keine einheitliche, allumfassende Definition von Religion gibt. Sie befasst sich mit mehr als nur einer gläubigen Verehrung eines Gottes, sondern ist ein hochkomplexes, vielschichtiges Phänomen. Diesem in der Vorlesung erwähnten Fazit stimme ich vollends zu, da es schier unmöglich scheint alle verschiedenen Interpretationen in einer einzigen Definition zu vereinen.\\