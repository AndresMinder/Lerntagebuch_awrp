\section{Vorlesung vom 09.04.2019}
Als heutiges Thema befassten wir uns mit dem \textit{Atheismus}. Dabei kam zuerst die Religionskritik und deren Formen zur Sprache. Es gibt die interne, interreligiöse und die externe Religionskritik. Die Interne gab es schon immer. Zum Beispiel kritisierte Jesus die buchstabenhörige und ausgrenzende Auslegung der Tora und betonte den in ihr enthaltenen Geist der Menschenfreundlichkeit oder Buddha kritisierte das Kastenwesen und die Göttervorstellungen zugunsten einer egalitären Gesellschaft und einer experimentellen Wirklichkeitserfahrung. Allerdings wurde nur sehr selten der Gottesglaube an sich in Frage gestellt.\\

Die interreligiöse Religionskritik ergibt sich aus der wechselseitigen Konkurrenz der Religionen in der theologischen Kontroverse. Religionsgeschichtlich finden wir diese in spezifischer Weise bei den Religionen \textit{Christentum} und \textit{Islam}. \\

Die populäre Religionskritik im 21. Jahrhundert ist der \textit{neue Atheismus}\footnote{ich interpretiere diesen als externe Religionskritik}. Dieser stammt vor allem aus den USA durch die Besiedlung der USA durch Glaubensflüchtlinge, oder auch Pilgerväter, aus Europa. \\

Neu an dieser Art von Atheismus ist z. B. das Outing nach dem Vorbild der Homosexuellen, sowie die naturwissenschaftliche Argumentation mit einer naturwissenschaftlichen Religionstheorie, der Ethik, welche explizit wissenschaftlich begründet wird usw.\\

Noch erwähnen möchte ich \glqq \textit{The four horsemen}\grqq. Es geht um vier Vertreter des \textit{neuen Atheismus} und bezeichnen sich selbst als die vier Reiter. Damit referieren sie auf die vier apokalyptischen Reiter aus der Johannesoffenbarung Kap. 6.\\

\begin{itemize}
	\item Daniel Dennett\\
	\item Sam Harris\\
	\item Christopher Hitchens\\
	\item Richard Dawkins\\
\end{itemize}

Ich selbst kann mich mit diesem \textit{neuen Atheismus} nicht identifizieren. Es macht den Eindruck, dass diese Bewegung ein ähnliches Ziel verfolgt wie die Religionen. Einfach ohne Götter und transzendente Wirklichkeiten, sondern mittels Empirismus und Rationalismus. Es wird behauptet, dass alles existierende logisch erklärbar ist. Wie es auch John Lennox im Video sagte, beinhaltet dies auch einen Glauben. Nur bin ich auch der Meinung, dass der menschliche Verstand schlicht zu klein ist um alles logisch zu erfassen. Viele Dinge sind in unserer begrenzten Vorstellung unmöglich und somit unbeweisbar. Den schliesslich denken wir in Modellen, welche nur das erfassen, was wir kennen.\\