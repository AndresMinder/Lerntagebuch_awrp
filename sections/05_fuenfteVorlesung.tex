\section{Vorlesung vom 19.03.2019}
Zu dieser Vorlesung habe ich keine Notizen, da ich leider nicht anwesend war. Deshalb werden nur die Inhalte der Folien reflektiert. Behandelt wurde das Thema \textit{\glqq Einführung in den Islam\grqq}. \\

Was für mich interessant scheint, ist der Ruf \textit{\glqq All\={a}hu akbar\grqq}, welcher Übersetzt \textit{Gott ist gross (grösser)} bedeutet. In der heutigen Welt wird dieser Ruf von der Allgemeinheit aber meist einem Terrorakt zugeordnet. Grund dafür ist die Verwendung des Rufes von islamistischen\footnote{zumindest was von den Medien behauptet wird} Terrorgesellschaften vor einem Anschlag. \\

Der Islam wird in drei Grade unterteilt:
\begin{enumerate}
	\item 	Die ganze Schöpfung ist Gott
			hingegeben, insofern Gott alles
			kontrolliert\\
	\item 	Jeder, der sich Gott hingibt,
			praktiziert Islam. Deshalb war 					u.a. schon Abraham Muslim\\
	\item 	Erst im engsten Sinne bezeichnet
			Islam die Religion, wie sie durch
			Muhammad im Koran offenbart worden 			ist.\\
\end{enumerate}

Dabei werden auch die Begriffe \textit{Islam}, \textit{Muslim} und \textit{Salam} unterschieden. \textit{Islam} ist abstammend vom Verb \textit{aslama}, was soviel wie \textit{sich hingeben/unterwerfen} bedeutet. Eine Partizipialform davon ist der Begriff \textit{Muslim} und bedeutet, \textit{der, der sich Gott hingibt}. Er bezieht sich also eher auf ein Individuum, welches den Islam auslebt. \textit{Salam} beschreibt den Frieden, welcher durch die Hingabe erlangt wird.\\

Der Koran steht im Mittelpunkt des islamischen Glaubens und hat \textit{Gott} darin manifestiert. Er ist die Grundlage aller traditionellen islamistischen Kulturen, den seine Worte prägen die Gebete, sowie er auch essentiell für das Identitätskonstrukt der Muslimen ist. Ein muslimisches Sprichwort besagt:\\

\begin{center}
	\textit{\glqq Der Koran ist ein 				sprechendes Universum und das Universum 		ein schweigender Koran.\grqq}\\
\end{center}

Der Begriff \textit{Koran} stammt von \textit{qara's} ab, was soviel wie \textit{vortragen} bedeutet. Nach muslimischen Glauben ist der Koran eine Rezitation der vom Erzengel Gabriel an Muhammad eingegebenen Worten. Er besteht aus 114 Suren und ist mit Ausnahme der ersten Sure (die Fatiha) der Länge nach geordnet. Zudem sind inhaltlich viele biblische Geschichten verarbeitet. \\