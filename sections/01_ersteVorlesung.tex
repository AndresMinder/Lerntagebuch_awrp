\section{Vorlesung vom 19.02.2019}
Nach der ersten Vorlesung, zu Beginn des Lerntagebuchs mache ich mir grundlegende Gedanken und Überlegungen über die folgenden Leitfragen:\\
\begin{itemize}
\item Welche Erfahrungen habe ich mit Religion/Religionsgemeinschaften?\\
\item Was denke ich über Religion?\\
\item Was denke ich über Wissenschaft und Vernunft?\\
\item Worin unterscheiden sich Religion und Naturwissenschaft?\\
\item Wie sehe ich das Verhältnis Religion – aufgeklärte Welt bzw. Glaube – Vernunft?
\end{itemize}

\hrulefill

Die Religion, rsp. der Glaube hat mich eigentlich immer begleitet. Ich wurde getauft und bin in römisch-katholischem Glauben aufgewachsen. Allerdings lag bei uns nicht wirklich die Religion selbst im Vordergrund, sondern mehr der Glaube. Der Glaube daran, dass nach dem Tod nicht das Nichts, sondern der Himmel auf uns wartet. Dass die Verstorbenen über uns wachen und uns durch das Leben begleiten. Eine sehr angenehme Vorstellung, welche ich gerne weiterhin in mir tragen werde. Die Naturwissenschaften unterscheiden sich meiner Meinung nach hauptsächlich in der eindeutigen Beweisbarkeit von der Religion. \\

Die Vernunft selbst hat meiner Ansicht nach nicht viel mit der Wissenschaft an und für sich zu tun. Der Mensch neigt zum extremistischen Verhalten. Grundlegend galt die Vernunft im 17. \& 18. Jahrhundert dafür, kritische Ansichten gegenüber der kirchlichen Dogmen zu bilden\footnote{Natürlich nicht nur dem gegenüber, aber dieser Aspekt steht hier im Vordergrund}. Es wurde begonnen zu hinterfragen. Somit kam der Rationalismus und der Empirismus auf und den Wissenschaften konnte ohne Unterdrückung der Kirche nachgegangen werden, was zu der heutigen modernen Welt führte\footnote{Das Ganze ist in sehr groben Umfang geschrieben}. Die aus den Wissenschaften hervorgegangen technischen Systeme unterstützen die globalisierte Welt in großem Masse, wie auch den Menschen bei alltäglichen Arbeiten. Dem Menschen wird durch die Vollautomatisierung immer mehr abgenommen. Nur geht bei dieser Vollautomatisierung von allen Dingen ein signifikanter Bestandteil, welcher eigentlich erst zu dieser Welt geführt hat, verloren. Das (\textit{kritische}) Denken! \glqq Cogito ergo sum\grqq\, - \glqq Ich denke, also bin ich\grqq, René Descartes (1637). Wenn dem allgemeinen Volk das Denken abgenommen wird, existiert es dann auch noch in dieser Welt? Wo bleibt da die Vernunft?\footnote{Es sind eher pessimistische Gedankengänge, allerdings sind es Fragen, die mich durchaus beschäftigen}\\

Das Verhältnis von Religion und der aufgeklärten Welt sehe ich in einer Wechselwirkung. Durch die hoch technisierte Welt erwarte ich persönlich, dass viele Menschen in eine Art Existenzkrise fallen. Man fühlt sich nicht mehr nützlich und über die sozialen Netzwerken werden perfekte Leben präsentiert. Somit sehe ich die Religion als prädestiniert, dass die Menschen zu ihr \glqq flüchten\grqq, um wieder einen Sinn zu finden.\\