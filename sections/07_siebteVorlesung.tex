\section{Vorlesung vom 02.04.2019}
Das behandelte Hauptthema heute war \textbf{Naturwissenschaft und Religion}. Als erstes ist es sinnvoll, das Verhältnis dieser beiden zu definieren, rsp. welche Fragen die beiden versuchen zu erklären.\\

\paragraph{Naturwissenschaften}
Der Mensch versucht, die Welt zu verstehen. \textbf{Wie} funktionieren die Dinge und was sind die Gesetzmäßigkeiten der Natur?\\

\paragraph{Religion}
Der Mensch versucht seinen Platz in der Welt zu finden. Was ist der Sinn des Lebens? \textbf{Warum} geschehen Dinge?\\

Die für mich interessanten Aspekte in dieser Vorlesung waren die Konflikte, welche zwischen den Naturwissenschaften und der Religion entstanden. Früher waren diese beiden stark miteinander verwoben. Wurden allerdings Forschungen ohne Einwilligung der Kirche gemacht, ging die Kirche sehr makaber damit um\footnote{z.B. mit Verbrennung}. Es wurden einige renommierte Wissenschaftler mit ihren Bewusstseins veränderlichen Hypothesen kurz angeschnitten.\\


\begin{tabular}{llll}
	Nikolaus Kopernikus & (1473-1543) & $\rightarrow$ & offenes
	heliozentrisches Weltbild \\ 
	Galileo Galilei  & (1564-1642) & $\rightarrow$ & erste überzeugende Argumente für 			physikalische\\
 	 & & &  Realität des heliozentrischen Weltbildes \\
	Isaac Newton & (1643-1727) & $\rightarrow$ & 3 Axiome der Mechanik und
	Gravitationsgesetz \\ 
	Charles Darwin & (1809-1882) & $\rightarrow$ & Evolutionstheorie \\ 
\end{tabular} 

Es wurde explizit der Fall von Galileo Galilei betrachtet. Mit seinen Argumenten für das heliozentrische Weltbild wurde das Verhältnis zwischen der Kirche und der Naturwissenschaften in den Wurzeln vergiftet. Er wurde daraufhin von der Kirche verurteilt, was aber heute von der kath. Kirche als Fehler und Irrtum gesehen wird. Auch Darwins Evolutionstheorie ist voll umfänglich von der kath. Kirche anerkannt. \\

Erwähnenswert ist, dass die Naturwissenschaften und die Religionen zwei unterschiedliche Sichtweisen auf die Welt haben und somit auch verschiedene Fragen versuchen zu beantworten. Naturwissenschaften fragen nach dem \textbf{Wie} und Religionen nach dem \textbf{Warum}. Schon in früheren Einträgen in diesem Lerntagebuch wurden ähnliche Aspekte aufgegriffen. Die Beantwortung der unterschiedlichen Fragestellungen der beiden haben eine gewisse Komplexität. Während es bei den Religionen einen relativ grossen Interpretationspielraum der Antworten gibt und es keine eindeutige universelle Antwort auf die Fragen hat, sind die Antworten bei den Naturwissenschaften evidenzbasierend und müssen empirisch beweisbar sein. Dadurch ist es schwierig, beide miteinander zu vergleichen, da sie sich meiner Meinung nach nicht mit demselben befassen und eine andere Orientierung haben. Die Religion verfolgt keine quantifizierbare, logisch nachvollziehbare Antworten. Es sind komplementäre Sichtweisen der Wirklichkeit.